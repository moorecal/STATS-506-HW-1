% Options for packages loaded elsewhere
\PassOptionsToPackage{unicode}{hyperref}
\PassOptionsToPackage{hyphens}{url}
\PassOptionsToPackage{dvipsnames,svgnames,x11names}{xcolor}
%
\documentclass[
  letterpaper,
  DIV=11,
  numbers=noendperiod]{scrartcl}

\usepackage{amsmath,amssymb}
\usepackage{iftex}
\ifPDFTeX
  \usepackage[T1]{fontenc}
  \usepackage[utf8]{inputenc}
  \usepackage{textcomp} % provide euro and other symbols
\else % if luatex or xetex
  \usepackage{unicode-math}
  \defaultfontfeatures{Scale=MatchLowercase}
  \defaultfontfeatures[\rmfamily]{Ligatures=TeX,Scale=1}
\fi
\usepackage{lmodern}
\ifPDFTeX\else  
    % xetex/luatex font selection
\fi
% Use upquote if available, for straight quotes in verbatim environments
\IfFileExists{upquote.sty}{\usepackage{upquote}}{}
\IfFileExists{microtype.sty}{% use microtype if available
  \usepackage[]{microtype}
  \UseMicrotypeSet[protrusion]{basicmath} % disable protrusion for tt fonts
}{}
\makeatletter
\@ifundefined{KOMAClassName}{% if non-KOMA class
  \IfFileExists{parskip.sty}{%
    \usepackage{parskip}
  }{% else
    \setlength{\parindent}{0pt}
    \setlength{\parskip}{6pt plus 2pt minus 1pt}}
}{% if KOMA class
  \KOMAoptions{parskip=half}}
\makeatother
\usepackage{xcolor}
\setlength{\emergencystretch}{3em} % prevent overfull lines
\setcounter{secnumdepth}{-\maxdimen} % remove section numbering
% Make \paragraph and \subparagraph free-standing
\makeatletter
\ifx\paragraph\undefined\else
  \let\oldparagraph\paragraph
  \renewcommand{\paragraph}{
    \@ifstar
      \xxxParagraphStar
      \xxxParagraphNoStar
  }
  \newcommand{\xxxParagraphStar}[1]{\oldparagraph*{#1}\mbox{}}
  \newcommand{\xxxParagraphNoStar}[1]{\oldparagraph{#1}\mbox{}}
\fi
\ifx\subparagraph\undefined\else
  \let\oldsubparagraph\subparagraph
  \renewcommand{\subparagraph}{
    \@ifstar
      \xxxSubParagraphStar
      \xxxSubParagraphNoStar
  }
  \newcommand{\xxxSubParagraphStar}[1]{\oldsubparagraph*{#1}\mbox{}}
  \newcommand{\xxxSubParagraphNoStar}[1]{\oldsubparagraph{#1}\mbox{}}
\fi
\makeatother

\usepackage{color}
\usepackage{fancyvrb}
\newcommand{\VerbBar}{|}
\newcommand{\VERB}{\Verb[commandchars=\\\{\}]}
\DefineVerbatimEnvironment{Highlighting}{Verbatim}{commandchars=\\\{\}}
% Add ',fontsize=\small' for more characters per line
\usepackage{framed}
\definecolor{shadecolor}{RGB}{241,243,245}
\newenvironment{Shaded}{\begin{snugshade}}{\end{snugshade}}
\newcommand{\AlertTok}[1]{\textcolor[rgb]{0.68,0.00,0.00}{#1}}
\newcommand{\AnnotationTok}[1]{\textcolor[rgb]{0.37,0.37,0.37}{#1}}
\newcommand{\AttributeTok}[1]{\textcolor[rgb]{0.40,0.45,0.13}{#1}}
\newcommand{\BaseNTok}[1]{\textcolor[rgb]{0.68,0.00,0.00}{#1}}
\newcommand{\BuiltInTok}[1]{\textcolor[rgb]{0.00,0.23,0.31}{#1}}
\newcommand{\CharTok}[1]{\textcolor[rgb]{0.13,0.47,0.30}{#1}}
\newcommand{\CommentTok}[1]{\textcolor[rgb]{0.37,0.37,0.37}{#1}}
\newcommand{\CommentVarTok}[1]{\textcolor[rgb]{0.37,0.37,0.37}{\textit{#1}}}
\newcommand{\ConstantTok}[1]{\textcolor[rgb]{0.56,0.35,0.01}{#1}}
\newcommand{\ControlFlowTok}[1]{\textcolor[rgb]{0.00,0.23,0.31}{\textbf{#1}}}
\newcommand{\DataTypeTok}[1]{\textcolor[rgb]{0.68,0.00,0.00}{#1}}
\newcommand{\DecValTok}[1]{\textcolor[rgb]{0.68,0.00,0.00}{#1}}
\newcommand{\DocumentationTok}[1]{\textcolor[rgb]{0.37,0.37,0.37}{\textit{#1}}}
\newcommand{\ErrorTok}[1]{\textcolor[rgb]{0.68,0.00,0.00}{#1}}
\newcommand{\ExtensionTok}[1]{\textcolor[rgb]{0.00,0.23,0.31}{#1}}
\newcommand{\FloatTok}[1]{\textcolor[rgb]{0.68,0.00,0.00}{#1}}
\newcommand{\FunctionTok}[1]{\textcolor[rgb]{0.28,0.35,0.67}{#1}}
\newcommand{\ImportTok}[1]{\textcolor[rgb]{0.00,0.46,0.62}{#1}}
\newcommand{\InformationTok}[1]{\textcolor[rgb]{0.37,0.37,0.37}{#1}}
\newcommand{\KeywordTok}[1]{\textcolor[rgb]{0.00,0.23,0.31}{\textbf{#1}}}
\newcommand{\NormalTok}[1]{\textcolor[rgb]{0.00,0.23,0.31}{#1}}
\newcommand{\OperatorTok}[1]{\textcolor[rgb]{0.37,0.37,0.37}{#1}}
\newcommand{\OtherTok}[1]{\textcolor[rgb]{0.00,0.23,0.31}{#1}}
\newcommand{\PreprocessorTok}[1]{\textcolor[rgb]{0.68,0.00,0.00}{#1}}
\newcommand{\RegionMarkerTok}[1]{\textcolor[rgb]{0.00,0.23,0.31}{#1}}
\newcommand{\SpecialCharTok}[1]{\textcolor[rgb]{0.37,0.37,0.37}{#1}}
\newcommand{\SpecialStringTok}[1]{\textcolor[rgb]{0.13,0.47,0.30}{#1}}
\newcommand{\StringTok}[1]{\textcolor[rgb]{0.13,0.47,0.30}{#1}}
\newcommand{\VariableTok}[1]{\textcolor[rgb]{0.07,0.07,0.07}{#1}}
\newcommand{\VerbatimStringTok}[1]{\textcolor[rgb]{0.13,0.47,0.30}{#1}}
\newcommand{\WarningTok}[1]{\textcolor[rgb]{0.37,0.37,0.37}{\textit{#1}}}

\providecommand{\tightlist}{%
  \setlength{\itemsep}{0pt}\setlength{\parskip}{0pt}}\usepackage{longtable,booktabs,array}
\usepackage{calc} % for calculating minipage widths
% Correct order of tables after \paragraph or \subparagraph
\usepackage{etoolbox}
\makeatletter
\patchcmd\longtable{\par}{\if@noskipsec\mbox{}\fi\par}{}{}
\makeatother
% Allow footnotes in longtable head/foot
\IfFileExists{footnotehyper.sty}{\usepackage{footnotehyper}}{\usepackage{footnote}}
\makesavenoteenv{longtable}
\usepackage{graphicx}
\makeatletter
\newsavebox\pandoc@box
\newcommand*\pandocbounded[1]{% scales image to fit in text height/width
  \sbox\pandoc@box{#1}%
  \Gscale@div\@tempa{\textheight}{\dimexpr\ht\pandoc@box+\dp\pandoc@box\relax}%
  \Gscale@div\@tempb{\linewidth}{\wd\pandoc@box}%
  \ifdim\@tempb\p@<\@tempa\p@\let\@tempa\@tempb\fi% select the smaller of both
  \ifdim\@tempa\p@<\p@\scalebox{\@tempa}{\usebox\pandoc@box}%
  \else\usebox{\pandoc@box}%
  \fi%
}
% Set default figure placement to htbp
\def\fps@figure{htbp}
\makeatother

\KOMAoption{captions}{tableheading}
\makeatletter
\@ifpackageloaded{caption}{}{\usepackage{caption}}
\AtBeginDocument{%
\ifdefined\contentsname
  \renewcommand*\contentsname{Table of contents}
\else
  \newcommand\contentsname{Table of contents}
\fi
\ifdefined\listfigurename
  \renewcommand*\listfigurename{List of Figures}
\else
  \newcommand\listfigurename{List of Figures}
\fi
\ifdefined\listtablename
  \renewcommand*\listtablename{List of Tables}
\else
  \newcommand\listtablename{List of Tables}
\fi
\ifdefined\figurename
  \renewcommand*\figurename{Figure}
\else
  \newcommand\figurename{Figure}
\fi
\ifdefined\tablename
  \renewcommand*\tablename{Table}
\else
  \newcommand\tablename{Table}
\fi
}
\@ifpackageloaded{float}{}{\usepackage{float}}
\floatstyle{ruled}
\@ifundefined{c@chapter}{\newfloat{codelisting}{h}{lop}}{\newfloat{codelisting}{h}{lop}[chapter]}
\floatname{codelisting}{Listing}
\newcommand*\listoflistings{\listof{codelisting}{List of Listings}}
\makeatother
\makeatletter
\makeatother
\makeatletter
\@ifpackageloaded{caption}{}{\usepackage{caption}}
\@ifpackageloaded{subcaption}{}{\usepackage{subcaption}}
\makeatother

\usepackage{bookmark}

\IfFileExists{xurl.sty}{\usepackage{xurl}}{} % add URL line breaks if available
\urlstyle{same} % disable monospaced font for URLs
\hypersetup{
  pdftitle={STATS 506 HW 1},
  pdfauthor={Calder Moore},
  colorlinks=true,
  linkcolor={blue},
  filecolor={Maroon},
  citecolor={Blue},
  urlcolor={Blue},
  pdfcreator={LaTeX via pandoc}}


\title{STATS 506 HW 1}
\author{Calder Moore}
\date{}

\begin{document}
\maketitle


\section{STATS 506 Homework 1}\label{stats-506-homework-1}

\subsection{Problem 1 - Abalone Data}\label{problem-1---abalone-data}

\subsubsection{a)}\label{a}

\begin{Shaded}
\begin{Highlighting}[]
\CommentTok{\#a)}
\CommentTok{\#Set directory and load data}
\FunctionTok{setwd}\NormalTok{(}\StringTok{"C:/Users/moore/Desktop/School/Master/STATS 506/STATS{-}506{-}HW{-}1/abalone"}\NormalTok{)}

\NormalTok{abalone }\OtherTok{\textless{}{-}} \FunctionTok{read.csv}\NormalTok{(}\StringTok{"abalone.data"}\NormalTok{)}

\CommentTok{\#Rename columns}
\FunctionTok{colnames}\NormalTok{(abalone)[}\DecValTok{1}\SpecialCharTok{:}\DecValTok{9}\NormalTok{] }\OtherTok{\textless{}{-}} \FunctionTok{c}\NormalTok{(}\StringTok{"Sex"}\NormalTok{, }\StringTok{"Length"}\NormalTok{, }\StringTok{"Diameter"}\NormalTok{, }\StringTok{"Height"}\NormalTok{, }\StringTok{"Whole weight"}\NormalTok{, }\StringTok{"Shucked weight"}\NormalTok{, }\StringTok{"Viscera weight"}\NormalTok{, }\StringTok{"Shell weight"}\NormalTok{, }\StringTok{"Rings"}\NormalTok{)}
\end{Highlighting}
\end{Shaded}

\subsubsection{b)}\label{b}

\begin{Shaded}
\begin{Highlighting}[]
\CommentTok{\#b)}
\FunctionTok{table}\NormalTok{(abalone}\SpecialCharTok{$}\NormalTok{Sex)}
\end{Highlighting}
\end{Shaded}

\begin{verbatim}

   F    I    M 
1307 1342 1527 
\end{verbatim}

There are 1,307 females, 1,526 males, and 1,342 infants.

\subsubsection{c) \#1}\label{c-1}

\begin{Shaded}
\begin{Highlighting}[]
\CommentTok{\#c)}

\CommentTok{\#1)}
\CommentTok{\#Using covariance function between Rings and each weight}
\FunctionTok{cov}\NormalTok{(abalone}\SpecialCharTok{$}\NormalTok{Rings, abalone}\SpecialCharTok{$}\StringTok{\textasciigrave{}}\AttributeTok{Whole weight}\StringTok{\textasciigrave{}}\NormalTok{)}
\end{Highlighting}
\end{Shaded}

\begin{verbatim}
[1] 0.8549952
\end{verbatim}

\begin{Shaded}
\begin{Highlighting}[]
\FunctionTok{cov}\NormalTok{(abalone}\SpecialCharTok{$}\NormalTok{Rings, abalone}\SpecialCharTok{$}\StringTok{\textasciigrave{}}\AttributeTok{Shucked weight}\StringTok{\textasciigrave{}}\NormalTok{)}
\end{Highlighting}
\end{Shaded}

\begin{verbatim}
[1] 0.3014396
\end{verbatim}

\begin{Shaded}
\begin{Highlighting}[]
\FunctionTok{cov}\NormalTok{(abalone}\SpecialCharTok{$}\NormalTok{Rings, abalone}\SpecialCharTok{$}\StringTok{\textasciigrave{}}\AttributeTok{Viscera weight}\StringTok{\textasciigrave{}}\NormalTok{)}
\end{Highlighting}
\end{Shaded}

\begin{verbatim}
[1] 0.1781965
\end{verbatim}

\begin{Shaded}
\begin{Highlighting}[]
\FunctionTok{cov}\NormalTok{(abalone}\SpecialCharTok{$}\NormalTok{Rings, abalone}\SpecialCharTok{$}\StringTok{\textasciigrave{}}\AttributeTok{Shell weight}\StringTok{\textasciigrave{}}\NormalTok{)}
\end{Highlighting}
\end{Shaded}

\begin{verbatim}
[1] 0.2818386
\end{verbatim}

Whole weight has the greatest correlation with Rings, with a correlation
of 0.855.

\subsubsection{c) \#2}\label{c-2}

\begin{Shaded}
\begin{Highlighting}[]
\CommentTok{\#2)}
\FunctionTok{cov}\NormalTok{(abalone}\SpecialCharTok{$}\NormalTok{Sex }\SpecialCharTok{==} \StringTok{"F"}\NormalTok{, abalone}\SpecialCharTok{$}\StringTok{\textasciigrave{}}\AttributeTok{Whole weight}\StringTok{\textasciigrave{}}\NormalTok{)}
\end{Highlighting}
\end{Shaded}

\begin{verbatim}
[1] 0.0681564
\end{verbatim}

\begin{Shaded}
\begin{Highlighting}[]
\FunctionTok{cov}\NormalTok{(abalone}\SpecialCharTok{$}\NormalTok{Sex }\SpecialCharTok{==} \StringTok{"M"}\NormalTok{, abalone}\SpecialCharTok{$}\StringTok{\textasciigrave{}}\AttributeTok{Whole weight}\StringTok{\textasciigrave{}}\NormalTok{)}
\end{Highlighting}
\end{Shaded}

\begin{verbatim}
[1] 0.05960039
\end{verbatim}

\begin{Shaded}
\begin{Highlighting}[]
\FunctionTok{cov}\NormalTok{(abalone}\SpecialCharTok{$}\NormalTok{Sex }\SpecialCharTok{==} \StringTok{"I"}\NormalTok{, abalone}\SpecialCharTok{$}\StringTok{\textasciigrave{}}\AttributeTok{Whole weight}\StringTok{\textasciigrave{}}\NormalTok{)}
\end{Highlighting}
\end{Shaded}

\begin{verbatim}
[1] -0.1277568
\end{verbatim}

Being infant has the strongest correlation with Whole weight, with a
correlation of -0.128. It is the greatest correlation in terms of
magnitude, the correlation is more strongly negative than the others are
positive.

\subsubsection{c) \#3}\label{c-3}

\begin{Shaded}
\begin{Highlighting}[]
\CommentTok{\#3)}
\CommentTok{\#Find the observation with the most Rings, and pulling the values for each weight from that observation}
\NormalTok{max\_weights }\OtherTok{\textless{}{-}} \FunctionTok{c}\NormalTok{(abalone[}\FunctionTok{which.max}\NormalTok{(abalone}\SpecialCharTok{$}\NormalTok{Rings),][}\StringTok{"Whole weight"}\NormalTok{],}
\NormalTok{                 abalone[}\FunctionTok{which.max}\NormalTok{(abalone}\SpecialCharTok{$}\NormalTok{Rings),][}\StringTok{"Shucked weight"}\NormalTok{],}
\NormalTok{                 abalone[}\FunctionTok{which.max}\NormalTok{(abalone}\SpecialCharTok{$}\NormalTok{Rings),][}\StringTok{"Viscera weight"}\NormalTok{],}
\NormalTok{                 abalone[}\FunctionTok{which.max}\NormalTok{(abalone}\SpecialCharTok{$}\NormalTok{Rings),][}\StringTok{"Shell weight"}\NormalTok{])}

\NormalTok{max\_weights}
\end{Highlighting}
\end{Shaded}

\begin{verbatim}
$`Whole weight`
[1] 1.8075

$`Shucked weight`
[1] 0.7055

$`Viscera weight`
[1] 0.3215

$`Shell weight`
[1] 0.475
\end{verbatim}

Whole weight is 1.8075, Shucked weight is 0.7055, Viscera weight is
0.3215, Shell weight is 0.475

\subsubsection{c) \#4}\label{c-4}

\begin{Shaded}
\begin{Highlighting}[]
\CommentTok{\#4)}
\CommentTok{\#Count the observations with a viscera weight greater than shell weight, and divide by the total number of observations}
\NormalTok{weight\_prop }\OtherTok{\textless{}{-}} \FunctionTok{sum}\NormalTok{(abalone}\SpecialCharTok{$}\StringTok{\textasciigrave{}}\AttributeTok{Viscera weight}\StringTok{\textasciigrave{}} \SpecialCharTok{\textgreater{}}\NormalTok{ abalone}\SpecialCharTok{$}\StringTok{\textasciigrave{}}\AttributeTok{Shell weight}\StringTok{\textasciigrave{}}\NormalTok{)}\SpecialCharTok{/}\FunctionTok{nrow}\NormalTok{(abalone)}
\NormalTok{weight\_prop}
\end{Highlighting}
\end{Shaded}

\begin{verbatim}
[1] 0.0651341
\end{verbatim}

Only \textasciitilde6.5\% of the observations have a Viscera weight
larger than their Shell weight.

\subsubsection{d)}\label{d}

\begin{Shaded}
\begin{Highlighting}[]
\CommentTok{\#d)}
\CommentTok{\#Pull out the different weight values by sex}
\NormalTok{cor\_F }\OtherTok{\textless{}{-}} \FunctionTok{c}\NormalTok{(}\FunctionTok{cov}\NormalTok{(abalone}\SpecialCharTok{$}\NormalTok{Sex }\SpecialCharTok{==} \StringTok{"F"}\NormalTok{, abalone}\SpecialCharTok{$}\StringTok{\textasciigrave{}}\AttributeTok{Whole weight}\StringTok{\textasciigrave{}}\NormalTok{),}
           \FunctionTok{cov}\NormalTok{(abalone}\SpecialCharTok{$}\NormalTok{Sex }\SpecialCharTok{==} \StringTok{"F"}\NormalTok{, abalone}\SpecialCharTok{$}\StringTok{\textasciigrave{}}\AttributeTok{Shucked weight}\StringTok{\textasciigrave{}}\NormalTok{),}
           \FunctionTok{cov}\NormalTok{(abalone}\SpecialCharTok{$}\NormalTok{Sex }\SpecialCharTok{==} \StringTok{"F"}\NormalTok{, abalone}\SpecialCharTok{$}\StringTok{\textasciigrave{}}\AttributeTok{Viscera weight}\StringTok{\textasciigrave{}}\NormalTok{),}
           \FunctionTok{cov}\NormalTok{(abalone}\SpecialCharTok{$}\NormalTok{Sex }\SpecialCharTok{==} \StringTok{"F"}\NormalTok{, abalone}\SpecialCharTok{$}\StringTok{\textasciigrave{}}\AttributeTok{Shell weight}\StringTok{\textasciigrave{}}\NormalTok{))}

\NormalTok{cor\_M }\OtherTok{\textless{}{-}} \FunctionTok{c}\NormalTok{(}\FunctionTok{cov}\NormalTok{(abalone}\SpecialCharTok{$}\NormalTok{Sex }\SpecialCharTok{==} \StringTok{"M"}\NormalTok{, abalone}\SpecialCharTok{$}\StringTok{\textasciigrave{}}\AttributeTok{Whole weight}\StringTok{\textasciigrave{}}\NormalTok{),}
           \FunctionTok{cov}\NormalTok{(abalone}\SpecialCharTok{$}\NormalTok{Sex }\SpecialCharTok{==} \StringTok{"M"}\NormalTok{, abalone}\SpecialCharTok{$}\StringTok{\textasciigrave{}}\AttributeTok{Shucked weight}\StringTok{\textasciigrave{}}\NormalTok{),}
           \FunctionTok{cov}\NormalTok{(abalone}\SpecialCharTok{$}\NormalTok{Sex }\SpecialCharTok{==} \StringTok{"M"}\NormalTok{, abalone}\SpecialCharTok{$}\StringTok{\textasciigrave{}}\AttributeTok{Viscera weight}\StringTok{\textasciigrave{}}\NormalTok{),}
           \FunctionTok{cov}\NormalTok{(abalone}\SpecialCharTok{$}\NormalTok{Sex }\SpecialCharTok{==} \StringTok{"M"}\NormalTok{, abalone}\SpecialCharTok{$}\StringTok{\textasciigrave{}}\AttributeTok{Shell weight}\StringTok{\textasciigrave{}}\NormalTok{))}

\NormalTok{cor\_I }\OtherTok{\textless{}{-}} \FunctionTok{c}\NormalTok{(}\FunctionTok{cov}\NormalTok{(abalone}\SpecialCharTok{$}\NormalTok{Sex }\SpecialCharTok{==} \StringTok{"I"}\NormalTok{, abalone}\SpecialCharTok{$}\StringTok{\textasciigrave{}}\AttributeTok{Whole weight}\StringTok{\textasciigrave{}}\NormalTok{),}
           \FunctionTok{cov}\NormalTok{(abalone}\SpecialCharTok{$}\NormalTok{Sex }\SpecialCharTok{==} \StringTok{"I"}\NormalTok{, abalone}\SpecialCharTok{$}\StringTok{\textasciigrave{}}\AttributeTok{Shucked weight}\StringTok{\textasciigrave{}}\NormalTok{),}
           \FunctionTok{cov}\NormalTok{(abalone}\SpecialCharTok{$}\NormalTok{Sex }\SpecialCharTok{==} \StringTok{"I"}\NormalTok{, abalone}\SpecialCharTok{$}\StringTok{\textasciigrave{}}\AttributeTok{Viscera weight}\StringTok{\textasciigrave{}}\NormalTok{),}
           \FunctionTok{cov}\NormalTok{(abalone}\SpecialCharTok{$}\NormalTok{Sex }\SpecialCharTok{==} \StringTok{"I"}\NormalTok{, abalone}\SpecialCharTok{$}\StringTok{\textasciigrave{}}\AttributeTok{Shell weight}\StringTok{\textasciigrave{}}\NormalTok{))}

\CommentTok{\#Create the table with the weights split out by sex, and rename the rows and columns for easier reading}
\NormalTok{sex\_weight\_table }\OtherTok{\textless{}{-}} \FunctionTok{rbind}\NormalTok{(cor\_F, cor\_M, cor\_I)}
\FunctionTok{colnames}\NormalTok{(sex\_weight\_table) }\OtherTok{\textless{}{-}} \FunctionTok{c}\NormalTok{(}\StringTok{"Whole weight"}\NormalTok{, }\StringTok{"Shucked weight"}\NormalTok{, }\StringTok{"Viscera weight"}\NormalTok{, }\StringTok{"Shell weight"}\NormalTok{)}
\FunctionTok{rownames}\NormalTok{(sex\_weight\_table) }\OtherTok{\textless{}{-}} \FunctionTok{c}\NormalTok{(}\StringTok{"Female"}\NormalTok{, }\StringTok{"Male"}\NormalTok{, }\StringTok{"Infant"}\NormalTok{)}
\NormalTok{sex\_weight\_table}
\end{Highlighting}
\end{Shaded}

\begin{verbatim}
       Whole weight Shucked weight Viscera weight Shell weight
Female   0.06815640     0.02716934     0.01567647   0.01977180
Male     0.05960039     0.02694935     0.01280370   0.01580163
Infant  -0.12775680    -0.05411869    -0.02848017  -0.03557343
\end{verbatim}

\subsubsection{e)}\label{e}

\begin{Shaded}
\begin{Highlighting}[]
\CommentTok{\#Create subsets of the data that only include the certain sex pairs so that we can run the t{-}tests for each pair: F and M, F and I, and M and I.}

\NormalTok{abalone\_FM }\OtherTok{\textless{}{-}} \FunctionTok{subset}\NormalTok{(abalone, abalone}\SpecialCharTok{$}\NormalTok{Sex }\SpecialCharTok{\%in\%} \FunctionTok{c}\NormalTok{(}\StringTok{"F"}\NormalTok{, }\StringTok{"M"}\NormalTok{))}
\NormalTok{abalone\_FI }\OtherTok{\textless{}{-}} \FunctionTok{subset}\NormalTok{(abalone, abalone}\SpecialCharTok{$}\NormalTok{Sex }\SpecialCharTok{\%in\%} \FunctionTok{c}\NormalTok{(}\StringTok{"F"}\NormalTok{, }\StringTok{"I"}\NormalTok{))}
\NormalTok{abalone\_MI }\OtherTok{\textless{}{-}} \FunctionTok{subset}\NormalTok{(abalone, abalone}\SpecialCharTok{$}\NormalTok{Sex }\SpecialCharTok{\%in\%} \FunctionTok{c}\NormalTok{(}\StringTok{"M"}\NormalTok{, }\StringTok{"I"}\NormalTok{))}

\NormalTok{ttest\_FM }\OtherTok{\textless{}{-}} \FunctionTok{t.test}\NormalTok{(abalone\_FM}\SpecialCharTok{$}\NormalTok{Rings }\SpecialCharTok{\textasciitilde{}}\NormalTok{ abalone\_FM}\SpecialCharTok{$}\NormalTok{Sex)}
\NormalTok{ttest\_FI }\OtherTok{\textless{}{-}} \FunctionTok{t.test}\NormalTok{(abalone\_FI}\SpecialCharTok{$}\NormalTok{Rings }\SpecialCharTok{\textasciitilde{}}\NormalTok{ abalone\_FI}\SpecialCharTok{$}\NormalTok{Sex)}
\NormalTok{ttest\_MI }\OtherTok{\textless{}{-}} \FunctionTok{t.test}\NormalTok{(abalone\_MI}\SpecialCharTok{$}\NormalTok{Rings }\SpecialCharTok{\textasciitilde{}}\NormalTok{ abalone\_MI}\SpecialCharTok{$}\NormalTok{Sex)}

\NormalTok{ttest\_FM}
\end{Highlighting}
\end{Shaded}

\begin{verbatim}

    Welch Two Sample t-test

data:  abalone_FM$Rings by abalone_FM$Sex
t = 3.69, df = 2741.8, p-value = 0.0002286
alternative hypothesis: true difference in means between group F and group M is not equal to 0
95 percent confidence interval:
 0.1999174 0.6533201
sample estimates:
mean in group F mean in group M 
       11.12930        10.70269 
\end{verbatim}

\begin{Shaded}
\begin{Highlighting}[]
\NormalTok{ttest\_FI}
\end{Highlighting}
\end{Shaded}

\begin{verbatim}

    Welch Two Sample t-test

data:  abalone_FI$Rings by abalone_FI$Sex
t = 29.477, df = 2508.9, p-value < 2.2e-16
alternative hypothesis: true difference in means between group F and group I is not equal to 0
95 percent confidence interval:
 3.023380 3.454304
sample estimates:
mean in group F mean in group I 
      11.129304        7.890462 
\end{verbatim}

\begin{Shaded}
\begin{Highlighting}[]
\NormalTok{ttest\_MI}
\end{Highlighting}
\end{Shaded}

\begin{verbatim}

    Welch Two Sample t-test

data:  abalone_MI$Rings by abalone_MI$Sex
t = -27.194, df = 2857.9, p-value < 2.2e-16
alternative hypothesis: true difference in means between group I and group M is not equal to 0
95 percent confidence interval:
 -3.014995 -2.609451
sample estimates:
mean in group I mean in group M 
       7.890462       10.702685 
\end{verbatim}

The difference in rings between females and males was the smallest. Both
females and males had significantly different numbers of rings from
infants.

\subsection{Problem 2 - Food Expenditure
Data}\label{problem-2---food-expenditure-data}

\subsubsection{a)}\label{a-1}

\begin{Shaded}
\begin{Highlighting}[]
\CommentTok{\#Load data}
\FunctionTok{setwd}\NormalTok{(}\StringTok{"C:/Users/moore/Desktop/School/Master/STATS 506/STATS{-}506{-}HW{-}1"}\NormalTok{)}

\NormalTok{food }\OtherTok{\textless{}{-}} \FunctionTok{read.csv}\NormalTok{(}\StringTok{"food\_expenditure.csv"}\NormalTok{)}
\end{Highlighting}
\end{Shaded}

\subsubsection{b)}\label{b-1}

\begin{Shaded}
\begin{Highlighting}[]
\CommentTok{\#Rename variables}
\FunctionTok{colnames}\NormalTok{(food) }\OtherTok{\textless{}{-}} \FunctionTok{c}\NormalTok{(}\StringTok{"ID"}\NormalTok{, }\StringTok{"Age"}\NormalTok{, }\StringTok{"Household Members"}\NormalTok{, }\StringTok{"State"}\NormalTok{, }\StringTok{"Currency"}\NormalTok{, }\StringTok{"Total Expenditures"}\NormalTok{, }\StringTok{"Grocery Store Expenditures"}\NormalTok{, }\StringTok{"Dining Out Expenditures"}\NormalTok{, }\StringTok{"Misc Expenditures"}\NormalTok{, }\StringTok{"Dine Out Count"}\NormalTok{, }\StringTok{"Alcohol"}\NormalTok{, }\StringTok{"Assistance Program"}\NormalTok{)}
\end{Highlighting}
\end{Shaded}

\subsubsection{c)}\label{c}

\begin{Shaded}
\begin{Highlighting}[]
\CommentTok{\#Count observations before currency restriction}
\FunctionTok{nrow}\NormalTok{(food)}
\end{Highlighting}
\end{Shaded}

\begin{verbatim}
[1] 262
\end{verbatim}

\begin{Shaded}
\begin{Highlighting}[]
\CommentTok{\#Restrict to only USD}
\NormalTok{food\_USD }\OtherTok{\textless{}{-}} \FunctionTok{subset}\NormalTok{(food, food}\SpecialCharTok{$}\NormalTok{Currency }\SpecialCharTok{\%in\%} \FunctionTok{c}\NormalTok{(}\StringTok{"USD"}\NormalTok{))}

\CommentTok{\#Recheck observation count}
\FunctionTok{nrow}\NormalTok{(food\_USD)}
\end{Highlighting}
\end{Shaded}

\begin{verbatim}
[1] 230
\end{verbatim}

Number of observations decreased by 32.

\subsubsection{d-h)}\label{d-h}

\begin{Shaded}
\begin{Highlighting}[]
\CommentTok{\#d) Remove under 18 since they probably don\textquotesingle{}t have a great sense of home expenditures usually, and also remove the couple of people who are apparently 150 since that seems sort of unlikely}

\NormalTok{food\_clean }\OtherTok{\textless{}{-}} \FunctionTok{subset}\NormalTok{(food\_USD, food\_USD}\SpecialCharTok{$}\NormalTok{Age }\SpecialCharTok{\%in\%} \FunctionTok{c}\NormalTok{(}\DecValTok{18}\SpecialCharTok{:}\DecValTok{100}\NormalTok{))}

\CommentTok{\#e) Remove the observations with a state of XX}
\NormalTok{food\_clean }\OtherTok{\textless{}{-}} \FunctionTok{subset}\NormalTok{(food\_clean, }\SpecialCharTok{!}\NormalTok{food\_clean}\SpecialCharTok{$}\NormalTok{State }\SpecialCharTok{\%in\%} \FunctionTok{c}\NormalTok{(}\StringTok{"XX"}\NormalTok{))}

\CommentTok{\#f) In total expenditures, there is an observation with "\textasciitilde{}350" rather than just 350. There are some negative values in the expenditure variables as well, and I\textquotesingle{}ll remove them here but I\textquotesingle{}m not sure if it\textquotesingle{}s great practice unless we\textquotesingle{}re certain that they are the result of a data collection error.}
\NormalTok{food\_clean}\SpecialCharTok{$}\StringTok{\textasciigrave{}}\AttributeTok{Total Expenditures}\StringTok{\textasciigrave{}}\NormalTok{[}\FunctionTok{which}\NormalTok{(food\_clean}\SpecialCharTok{$}\StringTok{\textasciigrave{}}\AttributeTok{Total Expenditures}\StringTok{\textasciigrave{}} \SpecialCharTok{==} \StringTok{"\textasciitilde{}350"}\NormalTok{)] }\OtherTok{\textless{}{-}} \DecValTok{350}
\NormalTok{food\_clean }\OtherTok{\textless{}{-}} \FunctionTok{subset}\NormalTok{(food\_clean, food\_clean}\SpecialCharTok{$}\StringTok{\textasciigrave{}}\AttributeTok{Total Expenditures}\StringTok{\textasciigrave{}} \SpecialCharTok{\textgreater{}} \DecValTok{0}\NormalTok{)}
\NormalTok{food\_clean }\OtherTok{\textless{}{-}} \FunctionTok{subset}\NormalTok{(food\_clean, food\_clean}\SpecialCharTok{$}\StringTok{\textasciigrave{}}\AttributeTok{Grocery Store Expenditures}\StringTok{\textasciigrave{}} \SpecialCharTok{\textgreater{}} \DecValTok{0}\NormalTok{)}
\NormalTok{food\_clean }\OtherTok{\textless{}{-}} \FunctionTok{subset}\NormalTok{(food\_clean, food\_clean}\SpecialCharTok{$}\StringTok{\textasciigrave{}}\AttributeTok{Dining Out Expenditures}\StringTok{\textasciigrave{}} \SpecialCharTok{\textgreater{}} \DecValTok{0}\NormalTok{)}
\NormalTok{food\_clean }\OtherTok{\textless{}{-}} \FunctionTok{subset}\NormalTok{(food\_clean, food\_clean}\SpecialCharTok{$}\StringTok{\textasciigrave{}}\AttributeTok{Misc Expenditures}\StringTok{\textasciigrave{}} \SpecialCharTok{\textgreater{}} \DecValTok{0}\NormalTok{)}

\CommentTok{\#g) There are some entries in dine out count that seem excessive. Eating out 30 times in a week is crazy for example, so I\textquotesingle{}ll remove observations with a count of 20 and over, but there are conceivably some people who would do it. I again don\textquotesingle{}t necessarily think that removing outliers just because they\textquotesingle{}re outliers is the best practice.}

\NormalTok{food\_clean }\OtherTok{\textless{}{-}} \FunctionTok{subset}\NormalTok{(food\_clean, food\_clean}\SpecialCharTok{$}\StringTok{\textasciigrave{}}\AttributeTok{Dine Out Count}\StringTok{\textasciigrave{}} \SpecialCharTok{\textless{}} \DecValTok{20}\NormalTok{)}

\CommentTok{\#h)}
\FunctionTok{nrow}\NormalTok{(food\_clean)}
\end{Highlighting}
\end{Shaded}

\begin{verbatim}
[1] 116
\end{verbatim}

Only 116 observations left after cleaning compared to an original 262.

\subsection{Problem 3 - Collatz
Conjecture}\label{problem-3---collatz-conjecture}

\subsubsection{a)}\label{a-2}

\begin{Shaded}
\begin{Highlighting}[]
\CommentTok{\#\textquotesingle{} Function to find the next value in the Collatz Conjecture sequence}
\CommentTok{\#\textquotesingle{}}
\CommentTok{\#\textquotesingle{} @param integer }
\CommentTok{\#\textquotesingle{}}
\CommentTok{\#\textquotesingle{} @returns the next number in the Collatz Conjecture sequence. Gives errors requesting numeric and integer inputs if the user attempts a non numeric or negative. Rounds floats to nearest integer value smaller than it.}
\CommentTok{\#\textquotesingle{} @examples an input of 5 returns 16, of 16 returns 8 according to the rule}
\NormalTok{nextCollatz }\OtherTok{\textless{}{-}} \ControlFlowTok{function}\NormalTok{(integer)\{}
  \CommentTok{\#Check for numbers}
  \ControlFlowTok{if}\NormalTok{(}\FunctionTok{is.numeric}\NormalTok{(integer))\{}
    \CommentTok{\#check for positive numbers}
    \ControlFlowTok{if}\NormalTok{(integer }\SpecialCharTok{\textgreater{}} \DecValTok{0}\NormalTok{)\{}
      \CommentTok{\#Check for evens}
      \ControlFlowTok{if}\NormalTok{(}\FunctionTok{round}\NormalTok{(integer) }\SpecialCharTok{\%\%} \DecValTok{2} \SpecialCharTok{==} \DecValTok{0}\NormalTok{)\{}
\NormalTok{        next\_int }\OtherTok{\textless{}{-}} \FunctionTok{round}\NormalTok{(integer)}\SpecialCharTok{/}\DecValTok{2}
\NormalTok{      \}}
      \CommentTok{\#Check for odds}
      \ControlFlowTok{else} \ControlFlowTok{if}\NormalTok{(}\FunctionTok{round}\NormalTok{(integer) }\SpecialCharTok{\%\%} \DecValTok{2} \SpecialCharTok{!=} \DecValTok{0}\NormalTok{)\{}
\NormalTok{        next\_int }\OtherTok{\textless{}{-}} \DecValTok{3}\SpecialCharTok{*}\FunctionTok{round}\NormalTok{(integer) }\SpecialCharTok{+} \DecValTok{1}
\NormalTok{      \}}
      \FunctionTok{return}\NormalTok{(next\_int)}
\NormalTok{    \}}
    \ControlFlowTok{else}\NormalTok{\{}
      \FunctionTok{return}\NormalTok{(}\StringTok{"Input must be positive"}\NormalTok{)}
\NormalTok{    \}}
\NormalTok{  \}}
  \ControlFlowTok{else}\NormalTok{\{}
    \FunctionTok{return}\NormalTok{(}\StringTok{"Input must be numeric"}\NormalTok{)}
\NormalTok{  \}}
\NormalTok{\}}

\CommentTok{\#Test}
\FunctionTok{nextCollatz}\NormalTok{(}\DecValTok{5}\NormalTok{)}
\end{Highlighting}
\end{Shaded}

\begin{verbatim}
[1] 16
\end{verbatim}

\begin{Shaded}
\begin{Highlighting}[]
\FunctionTok{nextCollatz}\NormalTok{(}\DecValTok{16}\NormalTok{)}
\end{Highlighting}
\end{Shaded}

\begin{verbatim}
[1] 8
\end{verbatim}

\subsubsection{b)}\label{b-2}

\begin{Shaded}
\begin{Highlighting}[]
\NormalTok{collatzSequence }\OtherTok{\textless{}{-}} \ControlFlowTok{function}\NormalTok{(integer)\{}
\NormalTok{  seq }\OtherTok{\textless{}{-}} \FunctionTok{c}\NormalTok{(integer)}
\NormalTok{  next\_int }\OtherTok{\textless{}{-}} \FunctionTok{nextCollatz}\NormalTok{(integer)}
  \ControlFlowTok{while}\NormalTok{(next\_int }\SpecialCharTok{\textgreater{}} \DecValTok{1}\NormalTok{)\{}
\NormalTok{    seq }\OtherTok{\textless{}{-}} \FunctionTok{c}\NormalTok{(seq, next\_int)}
\NormalTok{    next\_int }\OtherTok{\textless{}{-}} \FunctionTok{nextCollatz}\NormalTok{(next\_int)}
    \CommentTok{\#Making sure it includes the final "1" to end the sequence}
    \ControlFlowTok{if}\NormalTok{(next\_int }\SpecialCharTok{==} \DecValTok{1}\NormalTok{)\{}
\NormalTok{      seq }\OtherTok{\textless{}{-}} \FunctionTok{c}\NormalTok{(seq, next\_int)}
      \ControlFlowTok{break}
\NormalTok{    \}}
\NormalTok{  \}}
  \FunctionTok{return}\NormalTok{(seq)}
\NormalTok{\}}

\CommentTok{\#Test}
\FunctionTok{collatzSequence}\NormalTok{(}\DecValTok{5}\NormalTok{)}
\end{Highlighting}
\end{Shaded}

\begin{verbatim}
[1]  5 16  8  4  2  1
\end{verbatim}

\begin{Shaded}
\begin{Highlighting}[]
\FunctionTok{collatzSequence}\NormalTok{(}\DecValTok{19}\NormalTok{)}
\end{Highlighting}
\end{Shaded}

\begin{verbatim}
 [1] 19 58 29 88 44 22 11 34 17 52 26 13 40 20 10  5 16  8  4  2  1
\end{verbatim}

\subsubsection{c)}\label{c-5}

\begin{Shaded}
\begin{Highlighting}[]
\NormalTok{collatzList }\OtherTok{\textless{}{-}} \FunctionTok{c}\NormalTok{()}

\ControlFlowTok{for}\NormalTok{(i }\ControlFlowTok{in} \DecValTok{100}\SpecialCharTok{:}\DecValTok{500}\NormalTok{)\{}
\NormalTok{  collatzList }\OtherTok{\textless{}{-}} \FunctionTok{c}\NormalTok{(collatzList, }\FunctionTok{length}\NormalTok{(}\FunctionTok{collatzSequence}\NormalTok{(i)))}
\NormalTok{\}}
\FunctionTok{names}\NormalTok{(collatzList) }\OtherTok{\textless{}{-}} \FunctionTok{c}\NormalTok{(}\DecValTok{100}\SpecialCharTok{:}\DecValTok{500}\NormalTok{)}

\NormalTok{collatzList[}\FunctionTok{which}\NormalTok{(collatzList }\SpecialCharTok{==} \FunctionTok{max}\NormalTok{(collatzList))]}
\end{Highlighting}
\end{Shaded}

\begin{verbatim}
327 
144 
\end{verbatim}

\begin{Shaded}
\begin{Highlighting}[]
\NormalTok{collatzList[}\FunctionTok{which}\NormalTok{(collatzList }\SpecialCharTok{==} \FunctionTok{min}\NormalTok{(collatzList))]}
\end{Highlighting}
\end{Shaded}

\begin{verbatim}
128 
  8 
\end{verbatim}

327 gives the longest sequence with a length of 144, and 128 has the
shortest sequence, with a length of 8.




\end{document}
